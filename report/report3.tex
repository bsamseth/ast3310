\documentclass[11pt,twocolumn]{article}
\usepackage{preamble}

\bibliography{database}

\lhead{AST3310}
\chead{Term project 3: Title}
\rhead{Bendik Samseth}
\lfoot{}
\cfoot{}
\rfoot{\fancyplain{}{\thepage}}

\title{Term project 3: Title}
\author{Bendik \textsc{Samseth}}
\date{\today}
\begin{document}
  % make title page
\begin{titlepage}
  \newcommand{\HRule}{\rule{\linewidth}{0.5mm}}
  \center
  \textsc{\LARGE University of Oslo}\\[1.5cm]
  \textsc{\Large AST3310}\\[0.5cm]
  \textsc{\large Term Project 3}\\[0.5cm]
  \HRule \\[0.4cm]
  { \huge \bfseries Title}\\[0.4cm]
  \HRule \\[1.5cm]
  \Large \emph{Written by:}\\
  Bendik \textsc{Samseth}\\[3cm]
  {\large \today}\\[3cm]
  \vfill
\end{titlepage}


\onecolumn
  \tableofcontents
  \begin{abstract}

    In this report\ldots. All source material related to this report, which
    is cited at the relevant points, can be found at the projects
    GitHub repository~\cite{github}.

  \end{abstract}
%\end{@twocolumnfalse}
%]
\pagebreak
\twocolumn


\section{Introduction}

Hey hey, making a star, huh?


\section{Adding Convection}

Following form the last report, we know that the governing equations for a star is:

\begin{align}
  \pd{r}{m} &= \frac{ 1 }{ 4\pi r^2\rho }\label{eq:drdm}\\
  \pd{P}{m} &= - \frac{ Gm }{ 4\pi r^4 }\label{eq:dPdm}\\
  \pd{L}{m} &= \epsilon\label{eq:dLdm}\\
  \pd{T}{m} &= - \frac{ 3\kappa L }{ 256 \pi^2 \sigma r^4T^3 }\label{eq:dTdm}
\end{align}

Adding convective energy transport equates to changing the last equation. This is because the temperature gradient in Eq.~\eqref{eq:dTdm} is based on all the energy being moved out of the star through radiation. When convention kicks in, the temperature gradient will become smaller, because not all the energy needs to be transferred by radiation anymore.


\subsection{Deriving the new equation}

So, we want to find the new temperature gradient, including both convection and radiation. We still know that the total energy flux must be related to the luminosity:

\begin{align}
    F_R + F_C = \frac{L}{4\pi r^2}\label{eq:total-flux}
\end{align}

We still have the radiative flux, given as

\begin{align}
    F_R &= \frac{4acGmT^4}{3\kappa P r^2}\nabla \label{eq:radiative-flux}
\end{align}

We get Eq.~\eqref{eq:dTdm} by setting $F_C = 0$ and inserting into Eq.~\eqref{eq:total-flux}.

This time, we are not setting $F_C = 0$, obviously, so what should we set it to? The full derivation is given in~\cite{lecture-notes}, and is to involved to be repeted in this report. A quick summary will do.

We start from first principles, where we can say that $F_C = \rho c_P v \Delta T$, which says that the radiative flux is the energy $\rho c_P \Delta T$ a parcel of gas with density $\rho$ can move, with velocity $v$. The hard part is finding $v$ and the temperature difference. To this end, we will define three quantities, a shorthand symbol $\delta$\footnote{$\delta$ simplifies to 1 when we use the ideal gas law, see Appendix~\ref{app:nabla-ad}.}, the pressure scale height $H_P$\footnote{We get the expression for $H_P$ by using the assumption of hydrostatic equilibrium.}, and temperature gradient $\nabla $:

\begin{align}
    \delta &= -\left( \pd{\ln \rho}{\ln T} \right)_P=1\label{eq:delta-def}\\
    H_P &= -P\pd{r}{P} = \frac{P r^2}{\rho Gm} \label{eq:Hp-def}\\
    \nabla &= \pd{\ln T}{\ln P} = -\frac{H_P}{T}\pd{T}{r}\label{eq:grad-def}.
\end{align}

We want an equation for $\nabla$, as this would give us the temperature gradient we seek.

It can be shown (derivation once again in~\cite{lecture-notes}) that

\begin{align}
    v &= \sqrt{\frac{g\delta l_m^2}{4H_P}}(\nabla - \nabla^*)^{1/2}\label{eq:v}\\
    F_C &= \rho c_P T\sqrt{g\delta}H_P^{-\frac{3}{2}}\left(\frac{l_m}{2}\right)^2 (\nabla - \nabla^*)^{3/2}\label{eq:Fc-expression}
\end{align}

where $l_m$ is a constant with unit length. $\nabla^*$ is the temperature gradient for the little parcel of gas moving through the surroundings in the star. We only care about the actual $\nabla$, so we want to eliminate $\nabla^*$. But, before we can do that, we introduce two more $\nabla$'s, $\nabla_{ad}$ and $\nabla_{rad}$. These represent the temperature gradient if we had an adiabatic expansion of the parcel, and the temperature gradient needed to transport all the energy through radiation, respectively.

We can say that $F_C+F_R=F_{rad}$, meaning that the total flux from convection \emph{and} radiation is equal to the flux we would have if the temperature gradient was big enough for all energy to be moved by radiation (i.e. $\nabla_{rad}$). Let's insert our expressions.

\begin{align*}
    F_R + F_C &= \frac{4acGmT^4}{3\kappa P r^2}\nabla_{rad}\\
    F_C &= \frac{4acGmT^4}{3\kappa P r^2} (\nabla_{rad} - \nabla)\\
    (\nabla - \nabla^*)^{3/2} &= \frac{4acGmT^4}{3\kappa P r^2} \frac{(\nabla_{rad} - \nabla) }{\rho c_P T\sqrt{g\delta}H_P^{-\frac{3}{2}}}\\
    (\nabla - \nabla^*)^{3/2} &= \frac{64\sigma T^3}{3\kappa\rho^2 c_P l_m^2} \sqrt{\frac{H_P}{g\delta}} (\nabla_{rad} - \nabla) \\
    &\equiv \frac{U}{l_m^2} (\nabla_{rad} -\nabla)\numberthis\label{eq:5-8-answer}.
\end{align*}

where we have defined for convenience the constant $U$ (which represents an energy in this case):

\begin{align}
    U = \frac{64\sigma T^3}{3\kappa\rho^2 c_P} \sqrt{\frac{H_P}{g\delta}}\label{eq:U-def}.
\end{align}

To continue, we need to use one more magic formula~\cite[??]{lecture-notes},

\begin{align*}
    (\nabla^*-\nabla_{ad}) &= \frac{32\sigma T^3}{3\kappa \rho^2 c_p v}\frac{S}{dQ}(\nabla - \nabla^*) \\
    \equiv& \frac{32\sigma T^3}{3\kappa \rho^2 c_p v} l_m\,K (\nabla - \nabla^*)\numberthis\label{eq:5-69},
\end{align*}

where $K = S/l_m d Q$ is a geometric constant defined for convenience. If we  assume that the parcel is a sphere with diameter $d=l_m$, the parcel has a cross section $Q$ and surface area $S$, and $K$ reduces to

\begin{align}
    K = \frac{S}{l_m d Q} = \frac{4\pi r_p^2}{l_m^2 \pi r_p^2} = \frac{4}{l_m^2}.\label{eq:K-def}
\end{align}

Now we insert Eq.~\eqref{eq:5-69}, combined with Eq.~\eqref{eq:v} into the following equation:

\begin{align*}
    &(\nabla^* - \nabla_{ad}) = (\nabla - \nabla_{ad}) - (\nabla - \nabla^*)\\
    &\frac{32\sigma T^3}{3\kappa \rho^2 c_p v} l_m K (\nabla - \nabla^*) = (\nabla - \nabla_{ad}) - (\nabla - \nabla^*)\\
    &(\nabla - \nabla^*)\left[UK(\nabla-\nabla^*)^{-1/2} + 1\right]= \nabla - \nabla_{ad}\\
    &\Rightarrow \xi^2 + UK \xi - (\nabla-\nabla_{ad}) = 0\\
    &\Leftrightarrow \nabla = \xi^2 + UK \xi + \nabla_{ad}
\end{align*}

where $\xi = (\nabla-\nabla^*)^{1/2}$.

What we have is now an expression for $\nabla$, given by $\xi$ and $\nabla_{ad}$. The latter we know, so we need to find $\xi$. We do this by inserting $\xi$ into Eq.~\eqref{eq:5-8-answer}:

\begin{align*}
    (\nabla-\nabla^*)^{3/2} = \xi^3 = \frac{U}{l_m^2}(\nabla_{rad} - \nabla)\\
    \xi^3 = \frac{U}{l_m^2}\left(\nabla_{rad}  - (\xi^2 + UK \xi + \nabla_{ad})  \right)\\
    \Leftrightarrow \xi^3 + \frac{U}{l_m^2}\left( \xi^2 + UK\xi - (\nabla_{rad} - \nabla_{ad})\right) = 0\numberthis\label{eq:xi-third-order}.
\end{align*}

Eq.~\eqref{eq:xi-third-order} is a third order equation in $\xi$, but all the other parameters is knows, so we can in principle solve this equation, and thereby getting an expression for the true temperature gradient $\nabla$.

So, the steps we need to take are:

\begin{enumerate}
    \item Solve Eq.~\eqref{eq:xi-third-order} to get $\xi$.
    \item Insert into $\nabla = \xi^2 + UK\xi + \nabla_{ad}$.
    \item Solve for the gradient with respect to mass:
    \begin{align}
        \nabla &= \frac{P}{T}\pd{T}{P} = \frac{P}{T}\pd{m}{P}\pd{T}{m}\\
        &= \frac{P}{T} \left(- \frac{4\pi r^4}{Gm}\right) \pd{T}{m}\\
        \Rightarrow \pd{T}{m} &= -\frac{GmT}{4\pi r^4 P}\nabla.
    \end{align}
\end{enumerate}

\subsection{Instability criterion}
\label{sub:Instability criterion}

For our code, we would like to know if convection is taking place (and we need to use the harder formula derived in the last section), or if we can just use the old Eq.~\eqref{eq:dTdm}.

Convection will start if the temperature gradient $\partial T/\partial r$ is not large enough to transport all the energy by radiation. We say this by the inequality $\pd{T}{r} < \left(\pd{T}{r}\right)_{ad}$, or equvalently

\begin{align}
    \nabla_{rad} > \nabla_{ad}\label{eq:instability-criterion}.
\end{align}


So, the code will check if this is the case, and if so, apply the convection formula.

It should be noted that if this criteria is in fact not the case, and $\nabla_{rad} = \nabla_{ad}$, then $\xi$ will have the trivial solution of $\xi=0$, and consequently we end up with just the same equation as in Eq.~\eqref{eq:dTdm}. It is, however, still a good idea to skip the calculations if they are not necessary, so we keep the two cases separated.


\begin{appendices}
    \section{Calculating the Specific Heat for an Ideal Gas}
    In the following derivation, all lowercase quantities are per mass, unless otherwise stated.

    \subsection{Specific Heat at Constant Volume}
By definition, the specific heat at constant volume is:
\begin{align}
    dq &= c_v\,dT\hspace{0.5cm}(\text{constant volume})\label{eq:c_v-def}
\end{align}
Using the first law of thermodynamics, for constant volume, we get
\begin{align*}
    du &+ \frac{P\,dV}{m} = dq = c_v\,dT\\
    \Rightarrow c_v &= \frac{du}{dT}\numberthis\label{eq:c_v-expr}
\end{align*}

We find $u$ by using the equipartition theorem for a monoatomic, ideal gas. This gas has three degrees of freedom, so we get

\begin{align}
    u = \frac{3}{2}kT\cdot \frac{N}{m} = \frac{3}{2}kT\, \frac{1}{\mu m_u}\label{eq:u-equi}
\end{align}

Putting the last two equations together we get

\begin{align}
    c_v &= \frac{du}{dT} = \frac{3}{2} \frac{k}{\mu m_u}\label{eq:c-v-final}
\end{align}




\subsection{Specific Heat at Constant Pressure}
By definition, the specific heat at constant pressure is:
\begin{align}
    dq &= c_p\,dT\hspace{0.5cm}(\text{constant pressure})\label{eq:c_p-def}
\end{align}

We now remind our selfs of the ideal gas law, including the differential at constant pressure:

\begin{align}
    PV = NkT \Rightarrow P\,dV = Nk\,dT\label{eq:ideal-gas-diff}
\end{align}

We will use this in conjunction with the first law of thermodynamics:

\begin{align*}
    du &+ \frac{P\,dV}{m} = dq = c_p\,dT\\
    \Rightarrow c_p &= \frac{du}{dT} + \frac{Nk}{m} = \frac{du}{dT} + \frac{k}{\mu m_u}\\
    &= c_v + \frac{k}{\mu m_u}\numberthis\label{eq:c_v-c_p-rel}
\end{align*}

Finally, we insert the result from the last section:

\begin{align}
    c_p = \frac{5}{2} \frac{k}{\mu m_u}.\label{eq:c-p-final}
\end{align}



\section{Calculating the Adiabatic Temperature Gradient}
\label{app:nabla-ad}

The adiabatic temperature gradient is defined as follows:

\begin{align}
    \nabla_{ad} = \left(\pd{\ln T}{\ln P}\right)_{ad}\label{eq:nabla_ad-def}
\end{align}

It can be shown that this can be expressed as~\cite[Eq. (5.40)]{lecture-notes}

\begin{align}
    \nabla_{ad} = \frac{P\delta}{T\rho c_p}\label{eq:nabla_ad-expr}
\end{align}

The factor $\delta$ is defined in Eq.~\eqref{eq:delta-def}, and can be simplified significantly by using the ideal gas law. We write this up once more, this time with the density $\rho$:

\begin{align}
    P = \frac{\rho kT}{\mu m_u}\label{eq:ideal-gas-rho-def}
\end{align}

We then have

\begin{align*}
    \delta &= -\frac{T}{\rho}\pd{\rho}{T} = -\frac{T}{\rho}\left(-\frac{P\mu m_u}{kT}\frac{1}{T^2}\right)\\
    &= \frac{P\mu m_u}{\rho kT} = \frac{\rho kT}{\mu m_u} \frac{\mu m_u}{\rho kT}\\
    &= 1
\end{align*}

Now we get the adiabatic temperature gradient by applying Eq.~\eqref{eq:ideal-gas-rho-def} to Eq.~\eqref{eq:nabla_ad-expr}, along with the result for $c_p$ from the previous section:

\begin{align*}
    \nabla_{ad} &= \frac{P\delta}{T\rho c_p} = \frac{\rho kT}{\mu m_u} \frac{1}{T\rho c_p}\\
    &= \frac{k}{\mu m_u} \, \frac{2}{5}\frac{\mu m_u}{k} \\&= \frac{2}{5}
\end{align*}


\end{appendices}

\printbibliography
\end{document}
