\documentclass[11pt]{article}
\usepackage{preamble}

\lhead{AST3310}
\chead{Term project 2}
\rhead{Bendik Samseth}
\lfoot{}
\cfoot{}
\rfoot{\fancyplain{}{\thepage}}

\title{Term project 2: Modelling a Stellar Core}
\author{Bendik \textsc{Samseth}}
\date{\today}
\begin{document}
\maketitle


\begin{abstract}
In this report, this, that and more of this is discussed.
\end{abstract}

\section{Introduction}
When trying to understand the mechanics of a star the most logical
place to start is with what you can observe. Which properties of the
star can we measure rather easily? 

Well, we  can look at how much
radiation energy hits one particular spot on the earth, and by
assuming that the star radiates equally in all directions we can
calculate the total amount of energy sent out by the star, the
so-called luminosity (denoted as $L$). 

In addition, we can find the surface temperature (denoted $T$) by
looking at the color of the light emitted from the sun and using
Wien's displacements law. 

Lastly, we might be able to look at the orbits of all the planets and
do some number crunching and figure out the total mass of the star (denoted $M$).

But that's about it for what we can figure out by just looking at the
outside effects. There are many other properties we would like to know
about, e.g. the radius ($R$), density ($\rho$) and pressure ($P$). In
particular, we would like to have all of these properties as a
function of the radius, so that we can get an idea of what the cross
section of the star looks like. In order to achieve this, we need to
deploy many different branches of physics as well as a good number of
assumptions to simplify things for us. 


\section{The Governing Equations}
In this project, we will only be looking at the radiative core of the
star. This means that we will not look at the outer convection layer,
and assume that all energy transport is done through photon
radiation. For reference, this corresponds to the inner $\sim 70\,\%$ of the
sun. Furthermore, we will not consider time evolution; we look at one
particular moment in time. 

The following four differential equations govern the internal
structure of radiative core of our star\footnote{Derivations of the
  equations are not given, as they are shown in the lecture
  notes~\cite{lecture-notes}. A qualitative description of their
  meaning is given instead.}:

\begin{align}
  \pd{r}{m} &= \frac{ 1 }{ 4\pi r^2\rho }\label{eq:drdm}\\
  \pd{P}{m} &= - \frac{ Gm }{ 4\pi r^4 }\label{eq:dPdm}\\
  \pd{L}{m} &= \epsilon\label{eq:dLdm}\\
  \pd{T}{m} &= - \frac{ 3\kappa L }{ 256 \pi^2 \sigma r^4T^3 }\label{eq:dTdm}
\end{align}

The first thing to notice about these equations are that they are
differentials with respect to mass, and not radius. Remember that what
we want in the end is all the properties of the star as a function of
radius, so it would make sense to write the equations with respect to
$r$ instead\footnote{I will use lower case $r$ and $m$ when referring
  to the radius and mass as variables, and upper case $R$ and $M$ for
  the total radius and mass.}. It turns out that it is more numerically
stable to use $dm$ compared to $dr$, so we are going to treat the
radius as $r=r(m)$. When we plot the data later, however, we will
return to using $r$ as the variable. Eq. \eqref{eq:drdm} is simply
stating the relation between taking an infinitesimal step in $r$,
compared to an infinitesimal step in $m$. 

Eq. \eqref{eq:dPdm} is the assumption of hydrostatic equilibrium, and
states that if the gas in the star is to be at rest, then the outward
pressure must exactly balance the force of gravity acting on the gas. 

The $\epsilon$ in Eq. \eqref{eq:dLdm} represents the amount of
energy produced by nuclear fusion per time and mass. This quantity
will be treated more in a later section (ref??).

Eq. \eqref{eq:dTdm} is the temperature gradient need to transport all
the produced energy out of the sun. I should be noted that this is
only the correct temperature gradient when all the energy is
transported by radiation. 

\subsection{Equation of State}
In the equations~(\ref{eq:drdm}-\ref{eq:dTdm}) we use $T, \rho$ and
$P$. If we take a look at how many unknowns we have, we find that we
need one more equation. To this end, we are going to assume that we
can treat the gas in the star as an ideal gas, thus getting an
equation for the gas pressure through the ideal gas law:

\begin{align*}
  P_gV &= Nk_BT\\
  \Rightarrow P_g &= \frac{ N }{ V }k_BT = \frac{ \rho }{ \mu\,m_u }\label{eq:Pg}\numberthis,
\end{align*}
where $P_g$ is the gas pressure, $V$ is the volume, $N$ is the number
of particles and $T$ is the temperature of the gas.
The quantity $\mu\,m_u$ is the average mass of all particles in the
gas. $m_u$ is just a constant giving the average mass of a
nucleon. $\mu$ is slightly more complicated, and represents the mass
of the average particle in units of $m_u$ ($\mu$ is unit less). It
can be calculated a couple of ways. From the above equation we have

\begin{align}
  \mu &= \frac{\rho}{m_u}\frac{V }{ N } = \frac{ \rho }{ m_un_\text{tot} }
\end{align}
where $n_\text{tot}$ is the total number density of particles in the
gas. $n_\text{tot}$ can be taken as the sum of $n_i$ for all particles
$i$,

\begin{align}
  n_\text{tot} &= \sum_i n_i = n_e\sum_i \frac{ X_i\rho }{ C_i m_u}
\end{align}
where $X_i$ and $C_i$ is the mass fraction, and number of core
elements of particle $i$, respectively. For instance, $^4_2\text{He}$
gives $n_{^4_2\text{He}} = Y\rho/4m_u$, when $Y$ is the mass fraction
of Helium with four nucleons. The number density of the
electron, $n_e$ is written separate and should only consider the
number of \emph{free} electrons in the gas. In the case of only fully
ionized $^4_2\text{He}$, $n_e = 2n_{^4_2\text{He}}$ because each
ionized Helium core gives two electrons. I similar consideration
should be taken for each species present in the gas. 

Calculating $n_\text{tot}$ is not necessarily so straight forward, but
if we combine all the steps we have to take to compute $\mu$, we can
end up with a convenient formula:

\begin{align}
  \mu = \frac{1}{\sum (\text{particles provided per
  nucleus}\times\text{mass fraction}/\text{nucleons in core})}\label{eq:mu_0}.
\end{align}

This is the method used in the code, as a framework for summing over
all particles is central to the structure of the code base.


Let's now return to the original problem of finding an extra
equation. Eq. \eqref{eq:Pg} gives us an additional equation, but it
also introduces a new parameter, $P_g$. In addition to the gas
pressure, we have radiation pressure $P_{\text{rad}}$, so that the
total pressure is $P = P_g + P_\text{rad}$. Luckily, the expression
for radiation pressure is quite simple, and depends only on
temperature,
\begin{align}
  P_\text{rad} = \frac{ a }{ 3 }T^4
\end{align}
where $a = 4\sigma/c$ is a constant. This gives us finally the
additional equation of state we need;

\begin{align}
  P &= \frac{ \rho }{ \mu\,m_u }k_B T + \frac{ a }{ 3 }T^4\label{eq:P-eq_state}\\
  \Leftrightarrow \rho &= (P - \frac{a}{3}T^4) \frac{ \mu\,m_u }{ k_BT }\label{eq:rho-eq_state}.
\end{align}

I have listed the equation with respect to density as well; this
relation will be used to find $P$ given $\rho$ and $T$, and vice
versa.


\subsection{Energy Production}
In Eq. \eqref{eq:dLdm}, the quantity $\epsilon$ represents the amount
of energy produced per time and mass. We derive the value of
$\epsilon$ by looking at the reactions that produce energy inside a
star. It is given by
\begin{align}
  \epsilon = \sum_{ij}r_{ij}Q_{ij},
\end{align}
where $r_{ij}$ is the reaction rate for particles $i$ and $j$ (per
time and mass), and $Q_{ij}$ is the energy produced for each such
reaction. The exact values of $Q_{ij}$ can be found in the lecture
notes~\cite{lecture-notes}. More interesting is the rates, which are
given as

\begin{align}
  r_{ij} = \frac{n_in_j}{\rho(1+\delta_{ij})}\lambda_{ij}
\end{align}
where the symbols have their normal meaning (including the Kronecker
delta $\delta_{ij}$). $\lambda_{ij}$ is a complicated function of
temperature, and we will use tabulated expressions for these
functions, found once again in the lecture notes~\cite[Table
3.1]{lecture-notes}. 


There are a few different ways that fusion can happen, depending
on the conditions of the star. I our case we will simplify things a
bit and only consider the PPI and PPII chains (Proton-Proton based
fusion). 

One important thing to mention when calculating $\epsilon$ is that no
reaction should happen more often (have larger value for $r_{ij}$)
than the reaction(s) that produced the reactant(s) of the first
reaction. 

The last assumption we add is that all Deuterium produced by
Proton-Proton fusion, immediately fuses to $^3_2\text{He}$, thus
effectively merging the two first steps in the PP chains into one
reaction.


\section{Solving the Equations}
At this point we are ready to start solving the governing
equations. For simplicity, I have chosen to deploy the standard
Forward Euler method. One should then note that the simplicity comes
at the cost of an global error proportional to the step size, $dm$. We
will try to be smart about this step size in order to minimize the
effects of this. 

As a reminder, Forward Euler solves a differential equation $du/dx =
f(u,x)$ by 
\begin{align*}
  u_{i+i} = u_i + du = u_i + f(u_i,x_i)\,dx.
\end{align*}

In our case, with four differential equations, this corresponds to the
following:
\begin{align*}
  r_{i+1} &= r_i +  \frac{ 1 }{ 4\pi r^2\rho }\,dm\\
  P_{i+1} &= P_{i} - \frac{ Gm }{ 4\pi r^4 }\,dm\\
  L_{i+1} &= L_i + \epsilon\,dm\\
  T_{i+1} &= T_i - \frac{ 3\kappa L }{ 256 \pi^2 \sigma r^4T^3 }\,dm
\end{align*}



\printbibliography
\end{document}

%%% Local Variables:
%%% mode: latex
%%% TeX-master: t
%%% End:
