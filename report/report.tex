\documentclass[11pt,twocolumn]{article}
\usepackage{preamble}

\usepackage{wrapfig}
\usepackage{lscape}
\usepackage{rotating}
\usepackage{epstopdf}



\lhead{AST3310}
\chead{Term project 2: Modelling a Stellar Core}
\rhead{Bendik Samseth}
\lfoot{}
\cfoot{}
\rfoot{\fancyplain{}{\thepage}}

\title{Term project 2: Modelling a Stellar Core}
\author{Bendik \textsc{Samseth}}
\date{\today}
\begin{document}

\twocolumn[
\begin{@twocolumnfalse}
  \maketitle
  \tableofcontents
  \begin{abstract}

In this report, this, that and more of this is discussed. All source
material related to this report can be found at the projects GitHub
repository~\cite{github}.    

  \end{abstract}
\end{@twocolumnfalse}
]

\section{Introduction}
When trying to understand the mechanics of a star, the most logical
place to start is with what we can observe. Let's take a look at the
properties of a star that we can measure with relative ease. 

Firstly, we can look at how much radiation energy hits one particular
area, i.e. a solar panel on earth. By assuming that the star radiates
equally in all directions we can calculate the total amount of energy
sent out by the star, the so-called luminosity (denoted as $L$).

In addition, we can find the surface temperature (denoted $T$) by
looking at the color of the light emitted from the star and using
Wien's displacements law. 

Lastly, we might be able to look at the orbits of all the planets and
do some number crunching and figure out the total mass of the star
(denoted $M$). This last part may be easier said than done, for
instance in the case of a far away solar system where we can't even
see the planets orbiting, just their minuscule effect on the stars
orbit. However, for the purposes of this report, let's assume that we
are able to do so. 

But that's about it for what we can figure out by just looking at the
outside effects. There are many other properties we would like to know
about, e.g. the radius ($R$), density ($\rho$) and pressure ($P$) of
the star. In particular, we would like to have all of these properties
as a function of the radius, so that we can get an idea of what the
cross section of the star looks like. In order to achieve this, we
need to deploy many different branches of physics, including
thermodynamics, hydrostatics and nuclear physics. And of course, a good
number of assumptions to simplify things for us.


\section{The Governing Equations}
In this project, we will only be looking at the radiative core of the
star. This means that we will not look at the outer convection layer,
and assume that all energy transport is done through photon
radiation. For reference, this corresponds to the inner $\sim 70\,\%$ of the
sun. Furthermore, we will not consider time evolution; we look at one
particular moment in time. 

The following four differential equations govern the internal
structure of radiative core of our star\footnote{Derivations of the
  equations are not given, as they are shown in the lecture
  notes~\cite{lecture-notes}. A qualitative description of their
  meaning is given instead.}:

\begin{align}
  \pd{r}{m} &= \frac{ 1 }{ 4\pi r^2\rho }\label{eq:drdm}\\
  \pd{P}{m} &= - \frac{ Gm }{ 4\pi r^4 }\label{eq:dPdm}\\
  \pd{L}{m} &= \epsilon\label{eq:dLdm}\\
  \pd{T}{m} &= - \frac{ 3\kappa L }{ 256 \pi^2 \sigma r^4T^3 }\label{eq:dTdm}
\end{align}

The first thing to notice about these equations are that they are
differentials with respect to mass, and not radius. Remember that what
we want in the end is all the properties of the star as a function of
radius, so it would make sense to write the equations with respect to
$r$ instead\footnote{I will use lower case $r$ and $m$ when referring
  to the radius and mass as variables, and upper case $R$ and $M$ for
  the total radius and mass.}. It turns out that it is more numerically
stable to use $dm$ compared to $dr$, so we are going to treat the
radius as $r=r(m)$. When we plot the data later, however, we will
return to using $r$ as the variable. Eq.~\eqref{eq:drdm} is simply
stating the relation between taking an infinitesimal step in $r$,
compared to an infinitesimal step in $m$. 

Eq.~\eqref{eq:dPdm} is the assumption of hydrostatic equilibrium, and
states that if the gas in the star is to be at rest, then the outward
pressure must exactly balance the force of gravity acting on the gas. 

The $\epsilon$ in Eq.~\eqref{eq:dLdm} represents the amount of
energy produced by nuclear fusion per time and mass. This quantity
will be treated more in section~\ref{sec:energyproduction}.

Eq.~\eqref{eq:dTdm} is the temperature gradient need to transport all
the produced energy out of the star. I should be noted that this is
only the correct temperature gradient when all the energy is
transported by radiation. 

\subsection{Equation of State}
In the equations~(\ref{eq:drdm}-\ref{eq:dTdm}) we use $T, \rho$ and
$P$. If we take a look at how many unknowns we have, we find that we
need one more equation. To this end, we are going to assume that we
can treat the gas in the star as an ideal gas, thus getting an
equation for the gas pressure through the ideal gas law:

\begin{align*}
  P_gV &= Nk_BT\\
  \Rightarrow P_g &= \frac{ N }{ V }k_BT = \frac{ \rho }{ \mu\,m_u }\label{eq:Pg}\numberthis,
\end{align*}
where $P_g$ is the gas pressure, $V$ is the volume, $N$ is the number
of particles and $T$ is the temperature of the gas.
The quantity $\mu\,m_u$ is the average mass of all particles in the
gas. $m_u$ is just a constant giving the average mass of a
nucleon. $\mu$ is slightly more complicated, and represents the mass
of the average particle in units of $m_u$ ($\mu$ is unit less). It
can be calculated a couple of ways. From the above equation we have

\begin{align}
  \mu &= \frac{\rho}{m_u}\frac{V }{ N } = \frac{ \rho }{ m_un_\text{tot} }
\end{align}
where $n_\text{tot}$ is the total number density of particles in the
gas. $n_\text{tot}$ can be taken as the sum of $n_i$ for all particles
$i$,

\begin{align}
  n_\text{tot} &= \sum_i n_i = n_e+ \sum_i \frac{ X_i\rho }{ C_i m_u}
\end{align}
where $X_i$ and $C_i$ is the mass fraction, and number of core
elements of particle $i$, respectively. For instance, $^4_2\text{He}$
gives $n_{^4_2\text{He}} = Y\rho/4m_u$, when $Y$ is the mass fraction
of Helium with four nucleons. The number density of the
electron, $n_e$ is written separate and should only consider the
number of \emph{free} electrons in the gas. In the case of only fully
ionized $^4_2\text{He}$, $n_e = 2n_{^4_2\text{He}}$ because each
ionized Helium core gives two electrons. A similar consideration
should be taken for each species present in the gas. 

Calculating $n_\text{tot}$ is not necessarily so straight forward, but
if we combine all the steps we have to take to compute $\mu$, we can
end up with a convenient formula:

\begin{align}
  \mu = \frac{1}{\sum p_iX_i/C_i}\label{eq:mu_0}.
\end{align}
where $p_i$ is the number of particles provided per nucleus (i.e. the
core plus the number of ionized electrons).

This is the method used in the code, as a framework for summing over
all particles is central to the structure of the code base.


Let's now return to the original problem of finding an extra
equation. Eq.~\eqref{eq:Pg} gives us an additional equation, but it
also introduces a new parameter, $P_g$. In addition to the gas
pressure, we have radiation pressure $P_{\text{rad}}$, so that the
total pressure is $P = P_g + P_\text{rad}$. Luckily, the expression
for radiation pressure is quite simple, and depends only on
temperature,
\begin{align}
  P_\text{rad} = \frac{ a }{ 3 }T^4
\end{align}
where $a = 4\sigma/c$ is a constant. This gives us finally the
additional equation of state we need;

\begin{align}
  P &= \frac{ \rho }{ \mu\,m_u }k_B T + \frac{ a }{ 3 }T^4\label{eq:P-eq_state}\\
  \Leftrightarrow \rho &= (P - \frac{a}{3}T^4) \frac{ \mu\,m_u }{ k_BT }\label{eq:rho-eq_state}.
\end{align}

I have listed the equation with respect to density as well; this
relation will be used to find $P$ given $\rho$ and $T$, and vice
versa.


\subsection{Energy Production}
\label{sec:energyproduction}
In Eq.~\eqref{eq:dLdm}, the quantity $\epsilon$ represents the amount
of energy produced per time and mass. We derive the value of
$\epsilon$ by looking at the reactions that produce energy inside a
star. It is given by
\begin{align}
  \epsilon = \sum_{ij}r_{ij}Q_{ij},
\end{align}
where $r_{ij}$ is the reaction rate for particles $i$ and $j$ (per
time and mass), and $Q_{ij}$ is the energy produced for each such
reaction. The exact values of $Q_{ij}$ can be found in the lecture
notes~\cite{lecture-notes}, and also listed in the source
code~\cite[\texttt{reaction\_energies.h}]{github}. More interesting is the rates, which are given as

\begin{align}
  r_{ij} = \frac{n_in_j}{\rho(1+\delta_{ij})}\lambda_{ij}
\end{align}
where $n_i$ is the number density of particle $i$, and $\delta_{ij}$
is the Kronecker delta. $\lambda_{ij}$ is a complicated function of
temperature, and we will use tabulated expressions for these
functions, found once again in the lecture notes~\cite[Table
3.1]{lecture-notes}.


There are a few different ways that fusion can happen, depending
on the conditions of the star. In our case we will simplify things a
bit and only consider the PPI and PPII chains (Proton-Proton based
fusion). 

One important thing to mention when calculating $\epsilon$ is that no
reaction should happen more often (have larger value for $r_{ij}$)
than the reaction(s) that produced the reactant(s) of the first
reaction. For instance, in the PPII we have the two steps
\begin{align}
  ^3_2\text{He} + ^4_2\text{He}&\rightarrow ^7_4\text{Be}\\
  ^7_4\text{Be} + e^-&\rightarrow ^7_3\text{Li} + \nu_e.
\end{align}
Here, the second step can happen no more often than the first step; if
this was not the case, the Beryllium would quickly disappear and we
end up using Beryllium that does not exist. A similar considerations
should be made to all dependencies of each step in the cycles.


The last assumption we add is that all Deuterium produced by
Proton-Proton fusion, immediately fuses to $^3_2\text{He}$, thus
effectively merging the two first steps in the PP chains into one
reaction.


\section{Assumptions}
Up until this point, we've made several assumptions. In order to
remember the limitations of our results, I have summarized them in a
list:

\begin{itemize}
  \item No time evolution, we only look at one moment in time.
  \item The gas of the star behaves as an ideal gas.
  \item All energy transport is done through radiation.
  \item Energy production:
    \begin{itemize}
    \item Uniform and constant mass fractions. We use fixed values for 
      all the mass fractions, independent of radius, and assume that
      they do not change.
    \item Metals are not ionized at all, and all non-metals are fully
      ionized.
    \item Only the PPI and PPII reaction chains produce energy.
    \item Deuterium is produced and consumed at the same time,
      combining the first two steps of the PP chains into a single step.
    \end{itemize}
\end{itemize}



\section{Solving the Equations}
At this point we are ready to start solving the governing
equations. For simplicity, I have chosen to deploy the standard
Forward Euler method. One should then note that the simplicity comes
at the cost of an global error proportional to the step size, $dm$. We
will try to be smart about this step size in order to minimize the
effects of this. 

As a reminder, Forward Euler solves a differential equation $du/dx =
f(u,x)$ by the explicit scheme: 
\begin{align*}
  u_{i+i} = u_i + du = u_i + f(u_i,x_i)\,dx.
\end{align*}

In our case, with four differential equations, this corresponds to the
following:
\begin{align*}
  r_{i+1} &= r_i +  \frac{ 1 }{ 4\pi r_i^2\rho_i }\,dm\\
  P_{i+1} &= P_{i} - \frac{ Gm_i }{ 4\pi r^4_i }\,dm\\
  L_{i+1} &= L_i + \epsilon\,dm\\
  T_{i+1} &= T_i - \frac{ 3\kappa L_i }{ 256 \pi^2 \sigma r_i^4T_i^3
            }\,dm\\
  m_{i+1} &= m_i + dm
\end{align*}
I added the updating of the mass $m(r_i)=m_i$ for clarity.


\subsection{Choosing a Suitable Step Size}
Whenever a numerical solution to a differential equation is attempted,
one has to think about what step size is suitable. In a perfect
world, we would like to set it to an infinitesimal number, but the
constraints of reality forces us to make a compromise between accuracy
and speed. We need to find a value for $dm$ that gives an accurate
enough result, but does so in a reasonable time. 

In order to evaluate the effect of different step sizes, figure
\ref{fig:dm-variation-no-dss} shows $r(m)$ calculated for a number of
different values of $dm$\footnote{The shape and values of $r(m)$ in
  figure \ref{fig:dm-variation-no-dss} are not important. The graph is
  only meant to illustrate the different step sizes. Analysis of the
  shapes and input parameters will be discussed in detail
  later.}. From the figure we can see a large variation in $r(m)$ for
the lower values of $dm$. The solution seems to converge when
$dm \geq M_0\e{-3}$, but even there it is not perfectly aligned. It
turns out that the code runs very fast anyway (in large part because
it is written in C++), so it is no problem to crank the step size down
to $dm= M_0\e{-5}$, and still finish in under a second.

\begin{figure*}[ht]
  \centering
  \includegraphics[width=\linewidth]{fig/dm_variation_no_dss.png}
  \caption{\label{fig:dm-variation-no-dss} Plots of $r(m)$ using a
    range of different values for $dm$. We see that we need quite a
    large number of steps in order for the solution to converge.}
\end{figure*}


This is quite neat, but we could still try to be a bit smarter about
this than just using brute force. The downside to setting a fixed value for $dm$ is that we might spend
a lot of computational time on a part of our function(s) that is very
stable, and on the other side, we might not have enough precision for
where the function(s) is very steep. The alternative is \emph{dynamic
  step size}, which involves defining the step size in such a way
that the value of the function(s) does not change by more than a fixed
percentage. In the case of a diff.eq. $du/dx = f(u,x)$, we define the
step size in the following way:

\begin{align}
  \frac{du}{u} &\leq p\Leftrightarrow \frac{f\,dx}{u} \leq p\\
  \Rightarrow dx &= \frac{pu}{f},
\end{align}
where $p$ is the fraction that the solution is allowed to change by
after one step. In our case we have more than one equation, so we
impose a similar requirement to all equations and then simply choose
the smallest required step size.

The only thing we then have to define is what the value of $p$ should
be. As previously stated, the code is quite fast so we can afford to
be quite strict. The code used in all subsequent figures is set to
allow a maximum change of $5 \permil$. Figure \ref{fig:dm_variation}
shows the same plot as figure \ref{fig:dm-variation-no-dss}, but this
time including dynamic step size as well. The figure shows that DSS
gives good results. 

\begin{figure*}[ht]
  \centering
  \includegraphics[width=\linewidth]{fig/dm_variation.png}
  \caption{\label{fig:dm_variation} Plots of $r(m)$ using a range of
    different values for $dm$, including the solution obtained when
    using dynamic step size.}
\end{figure*}


\section{Changing the Parameters}
In this numerical model, we have the option to change the initial
parameters to our liking. In particular, we would like to find the set
of initial parameters that give the most realistic model of an actual
star. To this end we are now going to look at how the solution is
affected by changing $R_0,M_0, P_0$ etc. (In the following, $u_{0,i}$
will denote the default initial condition of quantity $u$, as defined
in the exercise text. The defaults correspond to the values for the
Sun at the bottom of the convection layer.)

\subsection{Changing $R_0$}
In order to see the effects of changing the initial radius, figure
\ref{fig:R-variation} has been produced. From the figure we can
conclude with the following for increasing initial radius:

\begin{itemize}
  \item $m$ dies off faster
  \item $P$ becomes \emph{much} smaller
\item $L$ remains constant for much longer (the core becomes a smaller
  fraction of the star)
\item $T$ becomes smaller
\item $\rho$ is similar to $T$, only that $\rho$ becomes \emph{much}
  smaller with increasing $R_0$
\end{itemize}

\begin{figure*}[ht]
  \centering
  \includegraphics[width=\linewidth]{fig/R_variation.png}
  \caption{\label{fig:R-variation} Plots for all parameters of the star
  showing the effects of changing the initial radius $R_0$.}
\end{figure*}

These effects are all reasonable given an increase in the size of the
star. We can compare this to what we know happens to a star towards the
end of its life cycle, where it becomes a red giant. In that case, the
radius increases, causing the surface temperature to drop (thus
glowing more red). The core has no reason to expand or contract, so it
becomes a smaller fraction of the total radius. Finally, with a constant total mass, it seems reasonable
that the density would drop significantly, along with the pressure
(all in accordance with Eq.~\eqref{eq:P-eq_state}). 

% What about why mass drops faster??
% We can also understand why the mass drops of faster for larger $R_0$,
% by looking at the density plot in figure \ref{fig:R-variation}. We see
% that larger $R_0$ causes a flatter

\subsection{Changing $T_0$}
Figure \ref{fig:T-variation} shows a similar comparison for a change
in the initial temperature $T_0$. We conclude that for increasing
initial temperature we have that:

\begin{itemize}
  \item $L$ falls of more rapidly, causing the core to expand
  \item $m$ falls off more towards the center (larger core)
  \item $T$ is smaller relative to $T_0$, which hints at a more or
    less unchanged temperature graph.
\end{itemize}

The main change caused by increasing $T_0$ seems to be a larger
core. This makes sense; a higher temperature in the outer parts would
imply that the temperatures needed for fusion to happen, are present
further out in star.

The plots for $\rho$ and $P$ also seem to comply with a more centrally
dense star.

\begin{figure*}[ht]
  \centering
  \includegraphics[width=\linewidth]{fig/T_variation.png}
  \caption{\label{fig:T-variation} Plots for all parameters of the star
  showing the effects of changing the initial temperature $T_0$.}
\end{figure*}


\subsection{Changing $\rho_0$}
Figure \ref{fig:rho-variation} shows what happens to the solutions
when we vary the initial density. We conclude that for increasing
initial density we have that:

\begin{itemize}
\item $m$ falls off more rapidly
\item $\rho$ and $P$ becomes much smaller compared their initial
  values.
\end{itemize}

The absolute main thing we gather from this figure is that the mass
drops of more rapidly, which of course makes sense given a larger
density further out in the star. 

The other graphs are quite hard to extract anything else from given
that the solution becomes unphysical very fast for large
$\rho_0$'s. I will therefore not give much time to interpret the
result for i.e. $P$, because they might as well be caused by very
strange behaviours of other parameters. 

\begin{figure*}[ht]
  \centering
  \includegraphics[width=\linewidth]{fig/rho_variation.png}
  \caption{\label{fig:rho-variation} Plots for all parameters of the star
  showing the effects of changing the initial density $\rho_0$.}
\end{figure*}

\subsection{Changing $P_0$}
Due to the connection between $P,\rho$ and $T$ shown in
Eq.~\eqref{eq:P-eq_state}, the effects of changing $P_0$ is included
in the above sections. In fact, the code sets the initial value for
$P$ based on the values for $\rho$ and $T$. Producing a similar plot
for variations in $P_0$ would be equivalent of changing $\rho_0$ or
$T_0$, or some arbitrary combination of the two according to Eq.~\eqref{eq:P-eq_state}.


\subsection{Changing $M_0$}
Figure \ref{fig:M-variation} shows the solutions for a range of total
masses. For increasing total mass we conclude that:

\begin{itemize}
  \item Both $\rho$ and $P$ becomes \emph{much} bigger in the outer
    parts, but get caught up by the smaller $M_0$'s towards the core
  \item $L$ starts to decline earlier (core becomes bigger), but it's
    not clear how increasing $M_0$ effects where $L\rightarrow 0$
  \item Low $M_0$'s causes $m$ to fall of faster in the beginning and
    slower towards the center, and the opposite is true for higher $M_0$'s
\end{itemize}

\begin{figure*}[ht]
  \centering
  \includegraphics[width=\linewidth]{fig/M_variation.png}
  \caption{\label{fig:M-variation} Plots for all parameters of the star
  showing the effects of changing the total mass $M_0$.}
\end{figure*}


\section{Finding the most realistic model}

\begin{sidewaysfigure*}[ht]
  \centering
  \includegraphics[scale=0.5]{fig/default-param.png}
  \caption{\label{fig:default-param} The parameters of the star, all as a
  function of radius, using the default parameters (corresponding to
  the bottom of the convection layer in the Sun).}
\end{sidewaysfigure*}

\begin{sidewaysfigure*}[ht]
  \centering
  \includegraphics[scale=0.5]{fig/best-param.png}
  \caption{\label{fig:best-param} The parameters of the star, all as a
  function of radius, using the parameters found to give the best result.}
\end{sidewaysfigure*}


Our end goal with this project was to determine how the parameters of
the star evolves as a function of radius, given a set of initial
conditions. As seen, what we chose to set as these conditions can have
a great impact on the final solution. In particular, many of the
conditions we have tried yielded very unphysical results. There are
certain things we know need to happen at the center of the star,
namely that the mass and luminosity both need to be zero (for
$r\rightarrow 0$). We can therefore set out to find a combination of
parameters that best fulfills this requirement, using what we learned
from the previous section. 

By using the default parameters (which correspond to the values at the
bottom of the convection layer in the Sun), we get the results in
figure \ref{fig:default-param}. As we can see, with these initial
conditions the mass goes to zero way to fast, in comparison with the
luminosity and radius. Because of this, the graphs don't seem to look
the way we would expect them to. We do however, see that the mass and
luminosity goes down with radius, while density, pressure and
temperature goes up towards the center. These are all characteristics
that we expect from the governing equations.

These parameters obviously don't fit the model very well\footnote{One
  might think that parameters taken from the Sun should fit quite
  well, because after all they are real values for a real
  star. However, we must to remember that we have made many
  assumptions and simplifications, and so it would actually be quite
  remarkable if they were to fit. We expect our model to fit a star
  that behaves in the way we have assumed, and the differences we find
to our Sun are related to the missing features (and/or lack of
understanding) of the Sun.}, so we would like to find the parameters
that represent the most realistic model of a star. First of we see
that the mass needs to fall off slower than it does in figure
\ref{fig:default-param}. The simplest way to do this, according the
the results form sec ??, is to reduce the initial density. I actually
need to reduce this quite a bit in order for the other parameters to
have the time to go to zero. Then it's a matter of getting $r,m$ and
$L$ synced up as much as possible. The one additional criteria that we
have is that the core should not be to small, and we would like it to
be at least $\sim 10\%$ of the total radius. 

Through a large amount of trial and error, both manually and programmatically varying the inputs, I
have landed on the following inputs:

\begin{align}
\begin{split}
  R_0/R_{0,i} &= 1.3, \ \ M_0/M_{0,i}= 0.95\\ 
  \rho_0/\rho_{0,i} &= 0.01,\ \ \    T_0/T_{0,i} = 0.8
\end{split}
\end{align}

Once again, $u_{0,i}$ refers to the value of quantity $u$ at the
bottom of convection zone in the Sun.

It must be said that these values have not been found in a very
analytical way, and so they are not the ``perfect'' values. It is also
quite likely that there exists many combinations of parameters that
all yield realistic models. As seen in the previous sections, some
parameters causes the core to become bigger, while others make it
smaller, and so some combination of changes could yield similar
results. But, I have not found a way to do this more rigorously than
by trial and error, so a single, approximate solution must be
sufficient.

With that said, the inputs produce the plots in figure
\ref{fig:best-param}. We can see the following characteristics from
the figure:

\begin{itemize}
  \item $L$ is constant at max for $r>0.2\cdot R_0$, and goes rapidly
    to zero for $r<0.2 \cdot R_0$, meaning that we have a core
    reaching out to about $20\%$ of $R_0$.
  \item Both $\rho,P$ and $T$ increases significantly as we move
    towards the center.
  \item The mass changes very little in the beginning, due to the low
    density, and falls of much faster towards the core. 
\end{itemize}





\section{Remarks on the Development of the model}
In large part the project has been enjoyable to work on, and I feel
quite satisfied with how it has turned out. I have also taken this
project as an opportunity to improve my knowledge of the C++ language,
and to this end I have at certain points done what might be considered
to much work. By this I mean using GitHub and related services
(Travis CI, Code coverage reports, README file for the landing page
etc.), using the documentation tool Doxygen and setting up
(unit) tests with Google Test. 

Much of the extra work related to this project arose when trying to
debug various problems. Things that have caused more than the normal
amount of headache are:

\begin{itemize}
  \item Implementing bilinear interpolation of the opacity table.
  \item Getting the energy production correct.
  \item Reproducing the results for the default parameters.
  \item Many other minor things...
\end{itemize}

All of this accumulated in a suite of $\sim 20$ (unit)
tests~\cite[tests/]{github}. In addition, after several days trying to
reproduce the plots in the exercise, I rewrote the entire code into
Python, in the hopes of finding the reasons why I couldn't. To my
despair, this yielded the exact same result, so I was very much
relived when the plots in the exercise was discovered to be incorrect and updated
with results that I was able to reproduce. 


%\onecolumn
\printbibliography
\end{document}

%%% Local Variables:
%%% mode: latex
%%% TeX-master: t
%%% End:
