\documentclass[11pt]{article}
\usepackage{preamble}

\lhead{AST3310}
\chead{Term project 2}
\rhead{Bendik Samseth}
\lfoot{}
\cfoot{}
\rfoot{\fancyplain{}{\thepage}}

\title{Term project 2: Modelling a Stellar Core}
\author{Bendik \textsc{Samseth}}
\date{\today}
\begin{document}
\maketitle


\begin{abstract}
In this report, this, that and more of this is discussed.
\end{abstract}

\section{Introduction}
When trying to understand the mechanics of a star the most logical
place to start is with what you can observe. Which properties of the
star can we measure rather easily? 

Well, we  can look at how much
radiation energy hits one particular spot on the earth, and by
assuming that the star radiates equally in all directions we can
calculate the total amount of energy sent out by the star, the
so-called luminosity (denoted as $L$). 

In addition, we can find the surface temperature (denoted $T$) by
looking at the color of the light emitted from the sun and using
Wien's displacements law. 

Lastly, we might be able to look at the orbits of all the planets and
do some number crunching and figure out the total mass of the star (denoted $M$).

But that's about it for what we can figure out by just looking at the
outside effects. There are many other properties we would like to know
about, e.g. the radius ($R$), density ($\rho$) and pressure ($P$). In
particular, we would like to have all of these properties as a
function of the radius, so that we can get an idea of what the cross
section of the star looks like. In order to achieve this, we need to
deploy many different branches of physics as well as a good number of
assumptions to simplify things for us. 


\section{The Governing Equations}
In this project, we will only be looking at the radiative core of the
star. This means that we will not look at the outer convection layer,
and assume that all energy transport is done through photon
radiation. For reference, this corresponds to the inner $\sim 70\,\%$ of the
sun. Furthermore, we will not consider time evolution; we look at one
particular moment in time. 

The following four differential equations govern the internal
structure of radiative core of our star\footnote{Derivations of the
  equations are not given, as they are shown in the lecture
  notes~\cite{lecture-notes}. A qualitative description of their
  meaning is given instead.}:

\begin{align}
  \pd{r}{m} &= \frac{ 1 }{ 4\pi r^2\rho }\label{eq:drdm}\\
  \pd{P}{m} &= - \frac{ Gm }{ 4\pi r^4 }\label{eq:dPdm}\\
  \pd{L}{m} &= \epsilon\label{eq:dLdm}\\
  \pd{T}{m} &= - \frac{ 3\kappa L }{ 256 \pi^2 \sigma r^4T^3 }\label{eq:dTdm}
\end{align}

The first thing to notice about these equations are that they are
differentials with respect to mass, and not radius. Remember that what
we want in the end is all the properties of the star as a function of
radius, so it would make sense to write the equations with respect to
$r$ instead\footnote{I will use lower case $r$ and $m$ when referring
  to the radius and mass as variables, and upper case $R$ and $M$ for
  the total radius and mass.}. It turns out that it is more numerically
stable to use $dm$ compared to $dr$, so we are going to treat the
radius as $r=r(m)$. When we plot the data later, however, we will
return to using $r$ as the variable. Eq. \eqref{eq:drdm} is simply
stating the relation between taking an infinitesimal step in $r$,
compared to an infinitesimal step in $m$. 

Eq. \eqref{eq:dPdm} is the assumption of hydrostatic equilibrium, and
states that if the gas in the star is to be at rest, then the outward
pressure must exactly balance the force of gravity acting on the gas. 

The $\epsilon$ in Eq. \eqref{eq:dLdm} represents the amount of
energy produced by nuclear fusion per time and mass. This quantity
will be treated more in a later section (ref??).

Eq. \eqref{eq:dTdm} is the temperature gradient need to transport all
the produced energy out of the sun. I should be noted that this is
only the correct temperature gradient when all the energy is
transported by radiation. 




\printbibliography
\end{document}

%%% Local Variables:
%%% mode: latex
%%% TeX-master: t
%%% End:
